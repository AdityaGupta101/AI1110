\let\negmedspace\undefined
\let\negthickspace\undefined
\documentclass[journal,12pt,twocolumn]{IEEEtran}

\usepackage{cite}
\usepackage{amsmath,amssymb,amsfonts,amsthm}
\usepackage{algorithmic}
\usepackage{graphicx}
\usepackage{textcomp}
\usepackage{xcolor}
\usepackage{txfonts}
\usepackage{listings}
\usepackage{enumitem}
\usepackage{mathtools}
\usepackage{gensymb}
\usepackage[breaklinks=true]{hyperref}
\usepackage{tkz-euclide} % loads  TikZ and tkz-base
\usepackage{listings}


\DeclareMathOperator*{\Res}{Res}
\renewcommand\thesection{\arabic{section}}
\renewcommand\thesubsection{\thesection.\arabic{subsection}}
\renewcommand\thesubsubsection{\thesubsection.\arabic{subsubsection}}

\renewcommand\thesectiondis{\arabic{section}}
\renewcommand\thesubsectiondis{\thesectiondis.\arabic{subsection}}
\renewcommand\thesubsubsectiondis{\thesubsectiondis.\arabic{subsubsection}}

\hyphenation{op-tical net-works semi-conduc-tor}
\def\inputGnumericTable{}                                 %%

\lstset{
frame=single, 
breaklines=true,
columns=fullflexible
}





\newtheorem{theorem}{Theorem}[section]
\newtheorem{problem}{Problem}
\newtheorem{proposition}{Proposition}[section]
\newtheorem{lemma}{Lemma}[section]
\newtheorem{corollary}[theorem]{Corollary}
\newtheorem{example}{Example}[section]
\newtheorem{definition}[problem]{Definition}


\vspace{3cm}

\title{
%	\logo{
Problem 12.13.4.1
%	}
}
\author{Aditya Gupta$^{*}$% <-this % stops a space
\thanks{The author is a first year Biotechnology student at IIT Hyderabad}}

\begin{document}

\maketitle
\parindent8pt
The question given is from class 12th \textbf{NCERT}, and is the 1st question of 4th exercise of 13th chapter. The question is given as below:-\\
\textit{State which of the following are not the probability distributions of a random
variable. Give reasons for your answer}.

\begin{enumerate}
	\item \begin{tabular}{|c|c|c|c|} 
   \hline
   X & 0 & 1 & 2 \\ 
   \hline
   P(X) & 0.4 & 0.4 & 0.2 \\ 
   \hline
  \end{tabular}

   \item \begin{tabular}{|c|c|c|c|c|c|} 
   \hline
   X & 0 & 1 & 2 & 3 & 4 \\ 
   \hline
   P(X) & 0.1 & 0.5 & 0.2 & -0.1 & 0.3 \\ 
   \hline
  \end{tabular}

	\item \begin{tabular}{|c|c|c|c|} 
   \hline
   Y & -1 & 0 & 1 \\ 
   \hline
   P(Y) & 0.6 & 0.1 & 0.2 \\ 
   \hline
   \end{tabular}

  \item \begin{tabular}{|c|c|c|c|c|c|} 
   \hline
   X & 0 & 1 & 2 & 3 & 4 \\ 
   \hline
   P(Z) & 0.3 & 0.2 & 0.4 & 0.1 & 0.05 \\ 
   \hline
   \end{tabular}
  \end{enumerate}
  
  Solution of the problem:-
  The sum of all the probabilities in a probability distribution must be one. Mathematically: If the probability distribution of a random variable X is the system of numbers and is represented as:-
  
  \begin{tabular}{|c|c|c|c|c|} 
	\hline
	X & $x_1$ & $x_2$ & . & $x_n$ \\ 
	\hline
	P(X) & $p_1$ & $p_2$ & . & $p_n$ \\ 
	\hline
	\end{tabular}
	
	where, \(p_i>0, \sum_{i=1}^{n}(p_i) =1, i=1,2,3...n\)
  

\parindent6pt Hence for all cases the sum of P(X) should be equal to one.
\begin{enumerate}

	\item \(\sum_{i=0}^{2}(p_i) =0.4+0.4+0.2= 1,\) \\
  \emph{This means it is a probability distribution.}

   \item \(\sum_{i=0}^{4}(p_i) =0.1+0.5+0.2-0.1+0.3= 1,\) \\
  \emph{This means it is a probability distribution.}
  
   \item \(\sum_{i=-1}^{1}(p_i) =0.6+0.1+0.2= 0.9,\) \\
  \emph{This means it is NOT a probability distribution.}
  
   \item \(\sum_{i=0}^{4}(p_i) =0.3+0.2+0.4+0.1+0.05= 1.05,\) \\
  \emph{This means it is NOT a probability distribution.}
  
\end{enumerate}


\end{document}

