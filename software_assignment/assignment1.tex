\documentclass{article}
\usepackage{graphicx}

\title{Random Song Shuffler - Code Report}
\author{Aditya Gupta (BT22BTECH11001)}

\begin{document}

\maketitle

\section{Introduction}
The Random Song Shuffler is a Python program that plays a collection of audio files in a random order. It utilizes a user interface (UI) created using the Tkinter library. This report provides an overview of the code implementation, including details on the steps used to create the random song shuffler and a screenshot of the generated UI.

\section{Code Overview}
The code consists of the following main components:

\begin{enumerate}
  \item Importing the necessary libraries:
    \begin{itemize}
      \item \texttt{os} library: to interact with the operating system and access the audio files in a directory.
      \item \texttt{random} library: to shuffle the list of audio files.
      \item \texttt{tkinter} library: to create the UI.
      \item \texttt{messagebox} module from \texttt{tkinter}: to display messages.
      \item \texttt{playsound} library: to play the audio files.
    \end{itemize}
  
  \item Defining the audio directory:
    \begin{itemize}
      \item The \texttt{audio\_directory} variable is set to the path where the audio files are located. This path needs to be updated to the correct directory path.
    \end{itemize}
  
  \item Creating the song shuffler:
    \begin{itemize}
      \item The code starts by retrieving a list of audio files in the specified directory using the \texttt{os.listdir()} function. Only files with ".mp3" or ".wav" extensions are considered.
      \item The \texttt{random.shuffle()} function is used to randomly shuffle the list of audio files.
      \item The \texttt{current\_song\_index} variable keeps track of the currently playing song.
    \end{itemize}
  
  \item Defining the UI:
    \begin{itemize}
      \item The Tkinter library is used to create a main application window.
      \item Three buttons are created: "Play," "Stop," and "Next."
      \item Each button is associated with a specific function (\texttt{play\_song()}, \texttt{stop\_song()}, and \texttt{next\_song()}).
    \end{itemize}
  
  \item Button Functions:
    \begin{itemize}
    \item The \texttt{play\_song()} function plays the current song using the \texttt{playsound} library. It retrieves the path of the current song based on \\
      the \texttt{current\_song\_index}.
            \item The \texttt{stop\_song()} function displays a message box indicating that the song has been stopped.
      \item The \texttt{next\_song()} function increments the \texttt{current\_song\_index} and plays the next song if available. If there are no more songs, a message box displays the end of the playlist.
    \end{itemize}
  
  \item Running the UI:
    \begin{itemize}
      \item The \texttt{window.mainloop()} function runs the main event loop, allowing the UI to be displayed and interacted with.
    \end{itemize}
\end{enumerate}

\section{Screenshot of the UI}
\begin{figure}[htbp]
  \centering
  \includegraphics[width=0.6\textwidth]{hi.png}
  \caption{Screenshot of the Random Song Shuffler UI}
\end{figure}

\section{Conclusion}
The Random Song Shuffler is a Python program that uses Tkinter to create a user interface for playing a collection of audio files in a random order. This report provided an overview of the code implementation, including a step-by-step description and a screenshot of the generated UI.

Please note that the report assumes the code has been executed successfully and all necessary dependencies and file paths have been properly set up.

\end{document}
